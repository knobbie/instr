\section{Grunnleggende}
\subsection{Starte programmet}
\subsubsection*{Metode 1}
\begin{enumerate}
	\item Trykk på konfigruasjon på skrivebordet.
  \item Velg Importer konfigurasjon fra katalog og finn filen "ProsjektAIM2014\_X\_EDIT\_YYYY" der X er versjon, og YYYY er dato på formen ddmm.
  \item Trykk på resetknappen på RCU500 i AIM1000-skapet.
	\item Trykk på "StartAIM.bat" på skrivebordet.AIM OS vil starte og mange vinduer vil åpnes opp. Logg deretter inn med brukernavn: Simrad og passord: simrad. 
\end{enumerate}

\subsubsection*{Metode 2}
\begin{enumerate}
  \item Siden lab-pcen er konfigurert til prosjektet, kan man benytte seg av filen "StartAIM\_OS1.bat" som ligger på skrivebordet.
  \item Logg inn med brukernavn: Simrad og passord: simrad.
\end{enumerate}

\subsection{Operatørpanel}
For å sette 

\subsection{Starte systemet}
\subsubsection*{Normal start}
For å starte systemet, trykk Gruppestart i operatørpanelet. 
\subsubsection*{Start etter nødstopp}
\subsubsection*{Start etter manuell stopp}
Hvis modulen står i intern modus, må den startes internt. Står modulen i eksternmodus skal den startes fra operatørpanelet. 

\subsection{Stoppe systemet}
\subsubsection*{Gruppestopp}
For å stoppe hele systemet eksternt, benytt gruppestopp.
\subsubsection*{Manuell stopp}
Hvis motoren står i intern modus, kan man stoppe hver enkelt modul manuellt. 

