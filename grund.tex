\section{Grunnleggende}
\subsection{Starte programmet}
\subsubsection*{Metode 1}
\begin{enumerate}
	\item Trykk på konfigurasjon på skrivebordet.
  \item Velg Importer konfigurasjon fra katalog og finn filen ''ProsjektAIM2014\_X\_EDIT\_YYYY'' der X er versjon, og YYYY er dato på formen ddmm.
  \item Trykk på resetknappen på RCU500 i AIM1000-skapet.
	\item Trykk på ''StartAIM.bat'' på skrivebordet slik at programmet starter opp. Logg deretter inn med brukernavn: Simrad og passord: simrad. 
\end{enumerate}

\subsubsection*{Metode 2}
\begin{enumerate}
  \item Siden lab-pcen er konfigurert til prosjektet, kan man benytte seg av filen ''StartAIM\_OS1.bat'' som ligger på skrivebordet.
  \item Logg inn med brukernavn: Simrad og passord: simrad.
\end{enumerate}

\subsection{Operatørpanel}
For å sette 

\subsection{Starte systemet}
\subsubsection*{Normal start}
For starte systemet i gruppestart. Sørg for at modulene står i ekstern modus, og systemet i AUTO. 
\subsubsection*{Start etter manuell stopp}
Hvis modulen står i intern modus, må den startes internt. Står modulen i eksternmodus skal den startes fra operatørpanelet. 

\subsection{Stoppe systemet}
\subsubsection*{Gruppestopp}
For å stoppe hele systemet eksternt, benytt gruppestopp. Det er forutsatt at alle modulene står i ekstern modus, samt at systemet står i AUTO. 
\subsubsection*{Manuell stopp}
Hvis motoren står i intern modus, kan man stoppe hver enkelt modul manuelt. 

